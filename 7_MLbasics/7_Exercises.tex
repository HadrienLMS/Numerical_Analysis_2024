\documentclass{article}
\usepackage[utf8]{inputenc}
\usepackage{amsmath}
\usepackage[scale=0.8]{geometry}
\usepackage{amssymb}
\usepackage{array}
\usepackage{color}

\title{Introduction to Machine Learning \\[1ex] \large Numerical analysis - University of Luxembourg}
\author{Exercises}
\date{}

\begin{document}

\maketitle


\section*{Linear regression}

\noindent \textbf{Exercise 1.} \textit{Linear Regression for Housing Prices} \\

In this exercise, you will build a simple machine-learning model to predict house prices based on given features. Here is a small dataset for training and testing your model:

\[
\begin{array}{|c|c|c|c|c|c|c|c|}
\hline
\text{Size (m²)} & \text{Bedrooms} & \text{Bathrooms} & \text{Garage Size} & \parbox[c]{2cm}{\centering Age \\ (Years)} & 
\parbox[c]{2cm}{\centering Distance to\\ Center (km)} & \parbox[c]{2cm}{\centering Renovation\\ status} & \parbox[c]{2cm}{\centering Price\\ (\$1000s)} \\
\hline
50  & 1 & 1 & 0 & 30 & 10 & 0 & 398.5  \\
60  & 2 & 1 & 1 & 20 & 8  & 1 & 468.5  \\
70  & 2 & 2 & 1 & 50 & 15 & 0 & 474.0  \\
80  & 2 & 2 & 2 & 10 & 5  & 2 & 593.0  \\
90  & 3 & 2 & 1 & 25 & 7  & 1 & 621.5  \\
100 & 3 & 3 & 2 & 40 & 12 & 0 & 647.3  \\
110 & 3 & 2 & 2 & 5  & 4  & 2 & 751.3  \\
120 & 4 & 3 & 2 & 15 & 6  & 1 & 798.0  \\
130 & 4 & 3 & 2 & 45 & 13 & 0 & 791.9  \\
140 & 4 & 3 & 3 & 8  & 3  & 2 & 911.0  \\
\hline
\end{array}
\]
%
\textbf{Legend for Garage size:} \\
\texttt{0 = no garage, 1 = 1 car, 2 = more} \\
%
\textbf{Legend for Renovation Status:} \\
\texttt{0 = Old, 1 = Medium, 2 = New}
\\

\begin{itemize}

\item{\textbf{Load the data}: define the dataset as a Python dictionary and convert it to a Pandas DataFrame.}

\item{\textbf{Preprocess the data}: normalize the features to bring them to a similar scale.}

\item{\textbf{Split the data}: divide the dataset into training and testing sets (80\% train, 20\% test).}

\item{\textbf{Train a model}: use Linear Regression from scikit-learn to predict house prices based on the features.}

\item{\textbf{Evaluate the model}: calculate the Mean Squared Error (MSE) on the test set.}

\end{itemize}

\section*{Non Linear regression}
\noindent \textbf{Exercise 2.} \textit{Non-Linear Regression for Housing Prices} \\

In this exercise, you will predict housing prices using non-linear regression on the California Housing dataset available in scikit-learn.

Steps to follow:

\begin{enumerate}
\item Load the dataset using "data = fetch\_california\_housing()" and inspect it
\item Train and evaluate two models:
\begin{itemize}
\item Polynomial Regression: A simple non-linear extension of linear regression.
\item Decision Tree Regression: A flexible tree-based model.
\end{itemize}
\item Compare both models using Mean Squared Error (MSE).
\end{enumerate}



\end{document}
