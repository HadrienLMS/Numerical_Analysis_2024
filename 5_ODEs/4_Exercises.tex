\documentclass{article}
\usepackage[utf8]{inputenc}
\usepackage{amsmath}
\usepackage[scale=0.8]{geometry}
\usepackage{amssymb}
\usepackage{array}
\usepackage{color}

\title{Ordinary differential equations \\[1ex] \large Numerical analysis - University of Luxembourg}
\author{Exercises}
\date{}

\begin{document}

\maketitle

\noindent \textcolor{blue}{Instructions: The minimum requirement is Ex 1. and Ex 3. You can also re-implement the examples from the class. Make sure you take time to experiment and build intuition for the explicit and implicit Euler methods. Ex. 2 and Ex. 5 give some insights into the field of dynamical systems.}

\vspace{0.5cm}
\noindent \textbf{Exercise 1.} \textit{Validation of Euler methods} \\
Consider the initial value problem, called the logistic equation
$$ \frac{dP}{dt} = r P \left(1- \frac{P}{K}\right), \quad P(0)= P_0, $$
where the exact solution is given by
$$y(t)=\frac{K}{ 1+ \left(\frac{K-P_0}{P_0}\right) e^{-rt}}.$$
The equation represents the evolution of a population size $P$ over time, where $K$ is the maximal population size, and $r$ is the growth rate.
\vspace{0.2cm} \\
Implement the explicit and implicit Euler method for this problem, and also a second order one-step method, for instance the explicit trapezoidal rule. Compare the accuracy of the methods and comment. Explain the difficulty you encounter when implementing the implicit Euler method. \\ 
As simulation parameters you can use $(P_0, r, K) = (10, 10, 200)$ and $(t_0, T) = (0, 2)$.

\vspace{0.5cm}
\noindent \textbf{Exercise 2.} \textit{An interesting pendulum} \\
 Let us consider the initial value problem, describing small undamped angular oscillations of a pendulum
 $$ y''(t) + \omega_0^2 y(t) = 0, \quad y(0)=y_0, \; y'(0) = y'_0. $$
 \underline{Implementation of ODE solvers}
 \begin{enumerate}
     \item Write the problem as a first order system
     \item For $\omega=1$, $(y_0, y'_0) = (-0.2, 0)$, a time step of $h=0.05$ and $T=20\pi$, solve the problem with the explicit Euler method and plot your solution. Plot the total energy over time
     $$E(t) =  \frac{1}{2}\left( y(t)'^2 + \omega_0^2y(t)^2 \right). $$
     Compare with the exact solution and comment. Do the same exercise with the implicit Euler method.
     \item For a system of the form 
     \begin{align*}
        x'(t) &= f(t,v) \\ 
        v'(t) &= g(t,x)
     \end{align*} 
     The symplectic Euler integrator reads
     \begin{align*}
         v_{n+1} &= v_n + h g_n(t_n, x_n) \\
         x_{n+1} &= x_n + h f_n(t_n, v_{n+1})
     \end{align*}
     Implement the scheme and look again at the conservation of energy. Compute experimentally the order of the scheme, and comment.
     \item Implement a 2nd and 4th order Runge-Kutta scheme. Are the methods energy conserving ? 
      \end{enumerate}
      
\noindent \underline{Phase flow}
    \begin{enumerate}
    \item Plot the numerical solution in the ``position-velocity'' phase space $(y(t), y'(t))$, and interpret the motion of the oscillator. Plot different solutions together using different initial conditions.
    \item Plot the phase flow of the system, that is the vector field $(y'(t), y''(t))$. Draw a parallel between the numerical solution and the phase flow.
 \end{enumerate}
 
 \noindent \underline{Non-linearity and damping} \\
 We add a damping parameter and non-linear term such as
 $$ y''(t) + 2 \zeta \omega_0 y'(t) + \omega_0^2 \sin(y(t)) = 0, \quad y(0)=y_0, y'(0) = y'_0. $$
 \begin{enumerate}
     \item Justify why the ODE is non-linear (\textit{Hint: use the superposition principle})
     \item Write the first order system for this new equation and adapt your previous implementation
     \item What is the effect of the parameter $\zeta$ ?
     \item When $y(t)$ is constant at all times, we say that we have reached an equilibrium point. What are the equilibrium points of the system ?
     \item Plot different trajectories in the phase space with the method of your choice and update the phase flow of the system. Comment on the solutions behaviour.
 \end{enumerate}
 
\vspace{0.5cm}
\noindent \textbf{Exercise 3.} \textit{Multistep methods} \\
Let us consider the IVP
\begin{align*}
    \frac{dy}{dt} = -y^2, \quad y(0) = y_0.
\end{align*}
\begin{itemize}
    \item Implement the Adams-Bashford (AB) methods for $r=1$ and $r=2$ and show the convergence rate of the methods. Initialize the methods thanks to the analytical solution that you will derive.
    \item Now initialize the methods with a one-step method of your choice. How is the convergence rate affected ?
 \end{itemize}
 \vspace{0.5cm}
\noindent \textbf{Exercise 4.} \textit{Multistep methods - predictor-corrector} \\
Using the same IVP as in the previous exercise,
\begin{itemize}
    \item Implement the following predictor-corrector method: given $y_i, f_i, f_{i-1}$, set
\begin{enumerate}
    \item $y_{i+1}^0=y_i+\frac{h}{2}\left(3 f_i-f_{i-1}\right)$,
    \item $f_{i+1}^0=f\left(t_{i+1}, y_{i+1}^0\right)$,
    \item $y_{i+1}=y_i+\frac{h}{2}\left(f_i+f_{i+1}^0\right)$,
    \item $f_{i+1}=f\left(t_{i+1}, y_{i+1}\right)$.
\end{enumerate}
Find the order of convergence of this method together with the Adams-Bashforth method for $r=2$ and the 2nd order RK method. Initialize the Adams-Bashforth and predictor-corrector methods with and then without the analytical solution, and explain the impact on the convergence order.
\item Compare the number of function evaluations per time step between the predictor-corrector method and the 2nd order RK method.
\item Do the same analysis for the 4th order RK and a 4th order predictor-corrector, given by
\begin{align*} 
y^{0}&=y_{n}+\frac{h}{24}\left( 55f_{n}-59f_{n-1}+37f_{n-2}-9f_{n-3}\right) \\
y^{p+1}&=y_{n}+\frac{h}{24}\left( 9f\left( t_{n+1},y^{p}\right)
+19f_{n}-5f_{n-1}+f_{n-2}\right) , \;p=0,1 \end{align*}
then set $y^{p+1} = y_{n+1}$
\end{itemize}

\vspace{0.5cm}
\noindent \textbf{Exercise 5.} \textit{Lorenz system} \\
Integrate the Lorenz atmospheric model
\begin{align*}
            x'(t) &= \sigma(y(t)-x(t)) \\
            y'(t) &= x(t)(\rho-z) - y(t) \\
            z'(t) &= x(t)y(t) - \beta z(t),
        \end{align*}
for the parameters: $\rho = 28$, $\sigma=10$, $\beta=8/3$, $(x_0,y_0,z_0) = (1,1,1)$, $T=40$ using the Euler method.
Compare your implementation with the \texttt{scipy solve\char`_ivp} integrator.
Perturb the initial condition such as $(x_0,y_0,z_0) = (1+10^{-5},1,1)$ and run the problem again. Comment your observations by looking at $x(t)$, $y(t)$ or $z(t)$. Is the system stable ?

\vspace{0.5cm}
\noindent \textbf{Exercise 6.} \textit{Spaceship trajectory}  \\
%The following classical example from astronomy gives a strong motivation to integrate initial value ODEs with local step size control.
Consider two bodies of (adimensional) mass $\mu = 0.012277471$ and $\hat{\mu} = 1 - \mu$ (earth and sun) in a planar motion, and a third body of negligible mass (your spaceship) moving in the same plane. The motion of the spaceship is governed by the equations
\begin{align*}
u_1^{\prime \prime} & =u_1+2 u_2^{\prime}-\hat{\mu} \frac{u_1+\mu}{D_1}-\mu \frac{u_1-\hat{\mu}}{D_2}, \\
u_2^{\prime \prime} & =u_2-2 u_1^{\prime}-\hat{\mu} \frac{u_2}{D_1}-\mu \frac{u_2}{D_2}, \\
D_1 & =\left(\left(u_1+\mu\right)^2+u_2^2\right)^{3 / 2}, \\
D_2 & =\left(\left(u_1-\hat{\mu}\right)^2+u_2^2\right)^{3 / 2} .
\end{align*}
Starting with the initial condition
\begin{align*}
& u_1(0)=0.994, u_2(0)=0, u_1^{\prime}(0)=0, \\
& u_2^{\prime}(0)=-2.00158510637908252240537862224,
\end{align*}
the solution is periodic with period $\approx 17.1$. Note that $D_1 = 0$ for $(u_1, u_2) = (-\mu,0)$ and $D_2 = 0$ for $(u_1, u_2) = (\hat{\mu},0)$, so we need to be careful when the orbit passes near these singularity points.
\begin{enumerate}
    \item Write the problem as a first order system, $U' = f(U)$, where $U =(u_1,u_1',u_2,u2')^T$,
    \item solve the ode with \texttt{scipy solve\char`_ivp} using an absolute error tolerance of $10^{-6}$ and relative error tolerance of $10^{-3}$ to integrate the problem on the interval
    $[0,17.1]$,
    \item how many time steps do you need ? what is the minimum and max time steps used by the solver ?
    \item Solve the problem with your own RK4 implementation. How many steps do you need to obtain a qualitatively correct orbit ?
    \item Comment and conclude on the usage of adaptive time stepping.
\end{enumerate}
\end{document}
