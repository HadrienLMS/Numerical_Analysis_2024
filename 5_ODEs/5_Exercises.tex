\documentclass{article}
\usepackage[utf8]{inputenc}
\usepackage{amsmath}
\usepackage[scale=0.8]{geometry}
\usepackage{amssymb}
\usepackage{array}
\usepackage{color}

\title{Ordinary differential equations \\[1ex] \large Numerical analysis - University of Luxembourg}
\author{Exercises}
\date{}

\begin{document}

\maketitle

\vspace{0.5cm}
\noindent \textbf{Exercise 1.} \textit{Truncation error (consistency)} \\
Consider the simplest initial value problem
$$u' = \lambda u, \quad \lambda \in \mathcal{R}$$
where the exact solution is given by
$$u(t) = e^{\lambda t} u(0) $$.  
%
Examine the ratio $r = U^{n+1}/U^{n}$ for the different numerical schemes 
\begin{itemize}
\item forward Euler
\item backward Euler
\item trapezoidal
\end{itemize}
In particular, plot the multiplier $r$ as a function of $\lambda k$ and compare it to the exact solution $r = e^{\lambda h}$, where $h = \Delta t$. 
 

\vspace{0.5cm}
\noindent \textbf{Exercise 2.} \textit{Logistic Equation - Euler methods} \\
Consider the initial value problem, called the logistic equation
$$ \frac{dP}{dt} = r P \left(1- \frac{P}{K}\right), \quad P(0)= P_0, $$
where the exact solution is given by
$$y(t)=\frac{K}{ 1+ \left(\frac{K-P_0}{P_0}\right) e^{-rt}}.$$
The equation represents the evolution of a population size $P$ over time, where $K$ is the maximal population size, and $r$ is the growth rate.
\vspace{0.2cm} \\
Implement the explicit and implicit Euler method for this problem, and also a second order one-step method, for instance the explicit trapezoidal rule. Compare the accuracy of the methods and comment. Explain the difficulty you encounter when implementing the implicit Euler method. \\ 
As simulation parameters you can use $(P_0, r, K) = (10, 10, 200)$ and $(t_0, T) = (0, 2)$.

\vspace{0.5cm}
\noindent \textbf{Exercise 3.} \textit{An interesting pendulum} \\
 Let us consider the initial value problem, describing small undamped angular oscillations of a pendulum
 $$ y''(t) + \omega_0^2 y(t) = 0, \quad y(0)=y_0, \; y'(0) = y'_0. $$
 \underline{Implementation of ODE solvers}
 \begin{enumerate}
     \item Write the problem as a first order system
     \item For $\omega=1$, $(y_0, y'_0) = (-0.2, 0)$, a time step of $h=0.05$ and $T=20\pi$, solve the problem with the explicit Euler method and plot your solution. Plot the total energy over time
     $$E(t) =  \frac{1}{2}\left( y(t)'^2 + \omega_0^2y(t)^2 \right). $$
     Compare with the exact solution and comment. Do the same exercise with the implicit Euler method.
     \item Implement a 2nd and 4th order Runge-Kutta scheme. Are the methods energy conserving ? 
      \end{enumerate}
      
\noindent \underline{Phase flow}
    \begin{enumerate}
    \item Plot the numerical solution in the ``position-velocity'' phase space $(y(t), y'(t))$, and interpret the motion of the oscillator. Plot different solutions together using different initial conditions.
    \item Plot the phase flow of the system, that is the vector field $(y'(t), y''(t))$. Draw a parallel between the numerical solution and the phase flow.
 \end{enumerate}
 
 \noindent \underline{Non-linearity and damping} \\
 We add a damping parameter and non-linear term such as
 $$ y''(t) + 2 \zeta \omega_0 y'(t) + \omega_0^2 \sin(y(t)) = 0, \quad y(0)=y_0, y'(0) = y'_0. $$
 \begin{enumerate}
     \item Justify why the ODE is non-linear (\textit{Hint: use the superposition principle})
     \item Write the first order system for this new equation and adapt your previous implementation
     \item What is the effect of the parameter $\zeta$ ?
     \item When $y(t)$ is constant at all times, we say that we have reached an equilibrium point. What are the equilibrium points of the system ?
     \item Plot different trajectories in the phase space with the method of your choice and update the phase flow of the system. Comment on the solutions behaviour.
 \end{enumerate}


\vspace{0.5cm}
\noindent \textbf{Exercise 4.} \textit{Spaceship trajectory}  \\
%The following classical example from astronomy gives a strong motivation to integrate initial value ODEs with local step size control.
Consider two bodies of (adimensional) mass $\mu = 0.012277471$ and $\hat{\mu} = 1 - \mu$ (earth and sun) in a planar motion, and a third body of negligible mass (your spaceship) moving in the same plane. The motion of the spaceship is governed by the equations
\begin{align*}
u_1^{\prime \prime} & =u_1+2 u_2^{\prime}-\hat{\mu} \frac{u_1+\mu}{D_1}-\mu \frac{u_1-\hat{\mu}}{D_2}, \\
u_2^{\prime \prime} & =u_2-2 u_1^{\prime}-\hat{\mu} \frac{u_2}{D_1}-\mu \frac{u_2}{D_2}, \\
D_1 & =\left(\left(u_1+\mu\right)^2+u_2^2\right)^{3 / 2}, \\
D_2 & =\left(\left(u_1-\hat{\mu}\right)^2+u_2^2\right)^{3 / 2} .
\end{align*}
Starting with the initial condition
\begin{align*}
& u_1(0)=0.994, u_2(0)=0, u_1^{\prime}(0)=0, \\
& u_2^{\prime}(0)=-2.00158510637908252240537862224,
\end{align*}
the solution is periodic with period $\approx 17.1$. Note that $D_1 = 0$ for $(u_1, u_2) = (-\mu,0)$ and $D_2 = 0$ for $(u_1, u_2) = (\hat{\mu},0)$, so we need to be careful when the orbit passes near these singularity points.
\begin{enumerate}
    \item Write the problem as a first order system, $U' = f(U)$, where $U =(u_1,u_1',u_2,u2')^T$,
    \item solve the ode with \texttt{scipy solve\char`_ivp} using an absolute error tolerance of $10^{-6}$ and relative error tolerance of $10^{-3}$ to integrate the problem on the interval
    $[0,17.1]$,
    \item how many time steps do you need ? what is the minimum and max time steps used by the solver ?
    \item Solve the problem with your own RK4 implementation. How many steps do you need to obtain a qualitatively correct orbit ?
    \item Comment and conclude on the usage of adaptive time stepping.
\end{enumerate}
\end{document}
